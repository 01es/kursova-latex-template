%% Вступ до дисертації
%% Приклад ненумерованого розділу
\chapter*{Вступ}

\paragraph{Актуальність теми}

Із розвитком та поширенням обчислювальної техніки в різних галузях діяльності людини об'єми даних, які зберігаються у файлах та базах даних, збільшуються високими темпами. Водночас користувачі, які працюють із цими даними, потребують кращих засобів отримання з них інформації. 

\paragraph{Мета і завдання дослідження}
Метою дослідження є розвиток\ldots. Для досягнення цієї мети були сформульовані та вирішені такі основні завдання:

\begin{itemize}
 \item провести порівняльний аналіз та дослідити ефективність\ldots;
 \item розробити методи та алгоритми, які вдосконалюють процес\ldots; 
 \item дослідити застосовність\ldots;
 \item розробити метод декомпозиції та алгоритм навчання\ldots;
 \item розробити математичне та програмне забезпечення\ldots
\end{itemize}

\subparagraph{Об'єкт дослідження}
Індуктивні методи самоорганізації моделей даних на основі карт Кохонена є об'єктом дослідження.

\emph{\textbf{Об'єктом дослідження}} є процеси\ldots

\emph{\textbf{Предметом дослідження}} є методи\ldots 

\paragraph{Структура дисертації}

Робота складається зі вступу, трьох розділів, висновків. Обсяг курсової 50~сторінок тексту, список використаних джерел містить 20~найменування).