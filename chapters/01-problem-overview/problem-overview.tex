%% problem-overview.tex  
% цей розділ присвячений огляду стану проблеми
\chapter{Огляд стану проблеми та основні поняття}\label{ch:01}

Моделювання даних завжди було невід'ємною частиною побудови інформаційних систем. Уважається, що інженери Young та Kent \cite{YoKe1958} першими висловили необхідність у чіткому та абстрактному способі специфікації інформаційних та часових характеристик проблем опрацювання даних. Значним поступом у розвитку моделювання даних, та інформаційних систем узагалі, стала праця дослідницької групи CODASYL (англ. COnference on DAta SYstems Language) \cite{CODASYL1962}. Важливим результатом CODASYL є розробка інформаційної алгебри \cite{Kal1983}, відповідно до якої, властивості об'єктів розглядають як відображення $p_k:E\rightarrow V_k$, де $E$ -- множина об'єктів предметної області, а $V_k$ -- множина значень властивості $p_k$. Об'єкти подаються у моделі впорядкованими значеннями його властивостей $p_1,\cdots,p_m$, які є координатами універсального інформаційного простору $V=V_1\times V_2\times\cdots\times V_m$. Ідеї інформаційної алгебри були, зокрема, використані в реляційній та семантичній моделях даних \cite{Oli2009}\ldots

\subsection{Підрозділ}

\subsection{Ще одни підрозділ}

\section{Висновки до розділу~\ref{ch:01}}